%% start of file `phuang_classic.tex'.
%% Copyright 2009 Peng Huang.
%

\documentclass[10pt]{moderncv}
\moderncvstyle{classic}

% character encoding
\usepackage[utf8]{inputenc}   % replace by the encoding you are using

% personal data
\firstname{Peng}
\familyname{Huang}
\title{Senior Software Engineer\dots}
\address{4702 Shiji Xingcheng\\Yangzhuang Nanlu Tongzhou Qu\\100100 Beijing, China}
\phone{+86 13717924071}
\email{shawn.p.huang@gmail.com}
\extrainfo{\weblink{http://shawnphuang.wordpress.com}}
% \photo[64pt]{phuang_picture}
% \quote{Any intelligent fool can make things bigger, more complex, and more violent. It takes a touch of genius -- and a lot of courage -- to move in the opposite direction.}

%\renewcommand{\listsymbol}{{\fontencoding{U}\fontfamily{ding}\selectfont\tiny\symbol{'102}}} % define another symbol to be used in front of the list items

% the ConTeXt symbol
\def\ConTeXt{%
  C%
  \kern-.0333emo%
  \kern-.0333emn%
  \kern-.0667em\TeX%
  \kern-.0333emt}

% slanted small caps (only with roman family; the sans serif font doesn't exists :-()
%\usepackage{slantsc}
%\DeclareFontFamily{T1}{myfont}{}
%\DeclareFontShape{T1}{myfont}{m}{scsl}{ <-> cork-lmssqbo8}{}
%\usefont{T1}{myfont}{m}{scsl}Testing the font

% command and color used in this document, independently from moderncv 
\definecolor{see}{rgb}{0.5,0.5,0.5} % for web links
\newcommand{\up}[1]{\ensuremath{^\textrm{\scriptsize#1}}}% for text subscripts
\definecolor{subsectioncolor}{rgb}{0.2,0.4,0.65}
\newcommand{\ssf}{\large\sffamily\mdseries\upshape}

%----------------------------------------------------------------------------------
%            content
%----------------------------------------------------------------------------------
\begin{document}
\maketitle
% \makequote
\section{Professional skills}
% \cvcomputer{OS}{Linux, Unix, Windows}{Developer}

\section{Education}
\cventry{1998--2002}{Bachelor of Computer Science}{Beijing Information Technical Institute}{}{}{}
% \cventry{1998--2002}{Bachelor of Computer Software}{Beijing Information Technical Institute}{}{}{1\up{st} year: 63\%\hspace{2em}2\up{nd} year: 76\%}

\section{Languages}
\cvlanguage{Mandarin}{Native}{}
\cvlanguage{English}{Very good}{I was working in US company for many years. Everyday I conmunicate with other team members in English.}

% \section{Master thesis}
% \cvitem{title}{\emph{On the design of modern curriculum vit\ae{}s}}
% \cvitem{supervisors}{Pr P. Picasso and Pr G. Klimt}
% \cvitem{description}{\small Study of the complex design of a curriculum vit\ae{}, also known as ``résumé''. In my opinion, a good design needs to be show the personality of its author. Some people will thus prefer a more classic style, and others will be more audacious\dots}

\section{Experience}
\\
\subsection{\ssf{Employments}}
%\subsection{\large\sffamily\mdseries\upshape{Employment}}
\cventry{Nov 2006--\\present}{Senior software engineer}{Red Hat}{aaa}{bbb}{ccc}
\cventry{May 2005--\\Nov 2006}{Senior software engineer \& Member of the Technology Working Group}{Motorola}{}{}{}
\cventry{July 2002--\\May 2005}{Senior software engineer \& Team leader}{Cass Hopen}{}{}{}
\\
\\
\subsection{\ssf{Projects}}

\section{Interests}
\cvitem{design}{\small I am a design fan, especially when it comes to typography and photography.}
\cvitem{adventure sports}{\small I like practicing adventure sports like skiing, rock climbing and scuba diving, and have been a boy scout for five years.}
\cvitem{travelling}{\small I have been living abroad during my childhood, and love travelling around the world.}

\section{Section with a list}
\cvlistitem{Single item.}
\cvlistitem{Another single item.}
\cvlistdoubleitem{Double\dots{}}{\dots{} item.}
\cvlistdoubleitem{Another double\dots{}}{\dots{} item.}

\section{Section with your own content}\closesection
Your content here, inside the normal \LaTeX{} environment. You can use any regular \LaTeX{} command, display mathematics
\[e =m\,c^2,\]
put some table or figure, \dots

\emptysection{}
\cvitem{Now}{Back to moderncv layout, without making a new section :-)}


\end{document}

%% end of file `phuang_classic.tex'.
